\documentclass[a4paper, 12pt]{article}
\usepackage{geometry}
\usepackage{setspace}
\usepackage{amsmath, amssymb}
\usepackage{amsthm}
\usepackage{thmtools}
\usepackage[utf8]{inputenc}
\usepackage[brazil]{babel}

\declaretheorem[style=definition, name=Definição]{definicao}

\hyphenation{com-ple-xi-da-de}
\hyphenation{ca-mi-nho}

\title{Relatório Final de Projeto}
\author{Plácido Francisco de Assis Andrade\\ 
	\texttt{placido.andrade@ufca.edu.br} 
	\and 
	Ikaro Ruan Penha Costa \\
	\texttt{ikaroruan@outlook.com}
	}
\date{}

\begin{document}
\maketitle


\section*{Identificação de Projeto}
\textbf{Título} \\
Análise Quantitativa de PPC's \\

\noindent \textbf{Edital de Referência} \\
PIBIC Edital XXXXXX \\

\section*{Área do Conhecimento Predominante}
Ciência da Computação

\onehalfspace
\section*{Resumo}

É certo que as Diretrizes Curriculares Nacionais regem os conteúdos que constituem os PPC's das Instituições de Ensino Superior do Brasil. Sendo assim, não 
é viável uma análise da pertinência de conteúdos de um curso de ensino superior, mas é possível proceder-se com um estudo do quão extenso ou profundo é a 
estrutura de um PPC. Logo, dois índices são propostos, os quais quantificam a complexidade e a possível retenção de alunos provenientes dos pré-requisitos da 
matriz curricular de um curso. Ainda, alguns parâmetros para construção dos índices são calculados e podem ser usados para a análise quantitativa do PPC, os quais 
foram indicados como sub-índices. Foi desenvolvido um aplicativo computacional de distribuição gratuita intitulado AnálisePPC como suporte para realização de 
simulações e comparação de matrizes curriculares semelhantes. Portanto, este projeto possibilitar tais índices com problemas acadêmicos enfrentados em diversos 
cursos, como evasão, abandono, repetência ou tracamento e simular possíveis simulações para diminuição da complexidade e retenção causada por pré-requisitos. \\

\textbf{Palavras-chave:} Estrutura curricular. Análise quantitativa. PPC. IES. AnálisePPC.

\section*{Introdução}

No processo de criação de um Projeto Político de Curso (PPC) a atenção prioritária é posta nos conteúdos a serem integralizados, sendo os maiores parâmetros 
observados as Diretrizes Curriculares Nacionais estabelecidas pelo Conselho Nacional de Educação (CNE) do Ministério da Educação. Esta é uma postura natual e 
não cabe, assim, observar a conveniência dos conteúdos propostos, mas torna-se interessante a análise quantitativa de fatores como a extensão do curso, a 
profundidade e a quantidade de cominhos de pré-requisitos da estrutura curricular, as cargas horárias complementares e de estágio obrigatório. \\

Para tal análise são propostos parâmetros, sub-índices, para a construção do dois índices propostos neste trabalho: Índice de Retenção e Índice de Complexidade. 
Tomando-se um turno de um estudante com carga horária diária de oito horas, sendo quatro horas despendidas em aulas teóricas e práticas e outras quatro em 
estudo individual, pode-se compor o sub-índice dos Turnos Efetivos como a quantidade de turnos exigidos por semestre em um PPC. A ideia principal do 
sub-índice do Peso dos Pré-requisitos considera a extensão dos caminhos de pré-requisitos e se estes estão em semestres consecutivos, visto que quanto mais 
longo um caminho e mais concentrado em semestre consecutivos, maior é o seu peso visando a integralização do curso. Ainda, quantificando a recorrência de um 
pré-requisito e diferentes caminhos, pode-se formar os Pré-requisitos Acumulados. Ao se unir tais parâmetos, pode-se construir o Índice de Complexidade, 
o qual considera os três parâmetros anteriores, e o Índice de Retenção que contempla a disposição das disciplinas ao longo dos semestres. \\

Como ferramenta para o cálculo de tais índices foi desenvolvido um programa computacional intitulado AnálisePPC. O seu algoritmo interpreta o PPC como 
um Grafo Acíclico Direcionado (\textit{DAG}, sigla em inglês), em que cada vértice é tido como uma disciplina e cada aresta indica a necessidade de integralizar 
uma disciplina para se poder cursar a disciplina subsequente. Dessa forma, pode-se percorrer o grafo em busca dos caminhos de pré-requisitos possíveis e, então, 
calcular os índices propostos. \\

Portanto, o programa AnálisePPC pode servir como uma útil ferramenta para os conselhos de cursos e de criação de PPC's das Instituições de Ensino Superior 
do Brasil. Possibilitando simulações por diversas alterações na estrutura curricular de curso de ensino superior, o AnálisePPC embasa as mudanças em complexidade e 
retenção do curso, afetando diretamente o discente, em decorrência da criação ou modificação de PPC's. Sendo assim, este projeto pode servir para 
busca de melhorias quanto aos problemas de evasão, repetência, abandono ou trancamento.

\section*{Justificativa}

Os autores do trabalho não têm ciência quanto a análises semelhantes da estrutura curricular de um PPC ou quanto ao programa em trabalhos 
prévios. Logo, a sua execução justifica-se por possibilitar além da análise quantitativa como ferramenta para criação e modificação de PPC's, 
mas também pode-se atribuir uma linha de pesquisa relacionada à ensino-aprendizagem na educação superior brasileira.

\section*{Referencial Teórico}
Como mencionado anteriormente, os autores não tem conhecimento de um trabalho anterior com análise semelhante de um PPC. Assim, os índices a serem apresentados 
a seguir foram propostos e refinados refinados pelos autores até a sua consolidação. \\ 

Por sua essencialidade na construção dos índices, os sub-índices Turnos Efetivos, Peso dos Pré-requitos e Pré-requisitos Acumulados terão sua formulação 
matemática apresentada. Em seguida, os Índices de Complexidade e Retenção serão definidos.

\subsection*{Turnos Efetivos ($\mathcal{T}_{ppc}$)}
Toma-se por aluno padrão aquele referente a interpretação de um estudante como um profissinal engajado em numa carga horária de 8 horas por dia, sendo 
4 horas despendidas em aulas teróricas e práticas e 4 horas de estudo individual. Considera-se, ainda, que o aluno padrão tem carga horária semestral de 
$T_0 = 320h$ de aulas teóricas e práticas distribuídas em 100 dias letivos. \\ 

Para a definição do Turno Efetivo serão utilizadas as gradezas apresentadas a seguir: \\

\begin{enumerate}
\item $n$: número de semestres propostos para integralização.
\item $M_{ppc}$: carga horária de integralização do PPC.
\item $M_{ac}$: carga horária de integralização das atividades complementares.
\item $M_{est}$: carga horária de estágio supervisionado.
\item $s_i$: semestre letivo $i$.
\item $M_i$: soma das cargas horárias das disciplinas alocadas no semestre $s_i$.
\end{enumerate}

\begin{definicao}
A quantidade de turno efetivo do semestre $s_i$ de um PPC com uma proposta de $n$ semestres  de integralização é \\

$$\tau_i = 2\frac{M_i}{T_0} + \frac{\frac{M_{ac}+M_{est}}{n}}{T_0}$$
\end{definicao}

\begin{definicao}
A quantidade de turnos efetivos do PPC com $n$ semestres letivos de integralização é \\
$$\mathcal{T}_{ppc} = \sum_{i = 1}^{n} \tau_i$$
\end{definicao}

\subsection*{Peso dos Pré-requisitos ($\mathcal{P}_{ppc}$)}

Inicialmente, deve-se conceituar um caminho de pré-requisitos. Dá-se à sequência $\alpha = \{ (m_i, s_i) \}_{i=1}^k$ a designação de um caminho de 
pré-requisitos sendo $m_i$ a carga horária e $s_i$ o semestre proposto da disciplina $d_i$. Atenta-se, pois, que a disciplina $(m_i, s_i)$ é um pré-requisito 
para $(m_{i+1}, s_{i+1})$. \\ 

O julgamento da dificuldade de uma disciplina é um processo subjetivo, não cabendo aqui a sua utilização. Logo, não é justo afirmar que uma disciplina de 
maior carga horária é de maior complexidade. Por outro lado, é possível assumir que uma disciplina alocada em um semestre mais avançado terá maior peso no 
tempo integralização previsto, haja vista que a perda de tal disciplina por algum motivo acarreta em menor tempo para recuperação do atraso causado.
Sendo assim, atribui-se o maior peso de um caminho $\alpha$ o valor do semestre $s_k$. \\

Ainda considerando o tempo necessário para recuperação do atraso causado pela perda de uma disciplina, considera-se que disciplinas consecutivas tenha o 
mesmo peso, pois a perda de uma delas resulta em uma reação em cadeia incrementando o valor dos seus semestres. Atribui-se, então, que todas as disciplinas 
consecutivas tem peso igual ao maior valor de semestre entre elas. \\

Isto significa que para duas disciplinas $(m_i, s_i)$ e $(m_j, s_j)$ presentes em um caminho $\alpha$, se $s_i - s_j = i - j$ então as duas disciplinas 
terão peso equivalentes de valor igual a $\max{\{ s_i, s_j \}}$. \\

Seja $\mathcal{S}_\alpha = {s_1, s_2, \ldots, s_k}$ o conjunto constituído pelos semestres do caminho $\alpha$. Define-se a relação de equivalência em 
$\mathcal{S}_\alpha$, 

$$ s_i \equiv s_j \qquad \text{se, e somente se, } \qquad s_i - s_j = i - j. $$

Para a definição do Peso dos Pré-requisitos, toma-se $Q_\alpha = \{ Q_{i_1}, Q_{i_2}, \ldots, Q_{i_p} \}$ de modo que $i_1 < i_2 < \cdots < i_p$ como 
o conjunto das classes de equivalências de $\mathcal{S}_\alpha$ e $s_{i_j} = \max Q_{i_j}$ para todo $j$ tal que $1 \leq j \leq p$. \\

Com efeito, sendo $\#Q_{i_j}$ a cardinalidade do conjunto $Q_{i_j}$ e $\log{x}$ o logaritmo na base 10, verifica-se

\begin{definicao}
Seja $\alpha$ um caminho de pré-requisitos, o Peso dos Pré-requisitos das disciplinas do caminho $\alpha$ é
$$ \mathcal{P}_\alpha = \sum_{j=1}^p (\#Q_{i_j}\log{s_{i_j}}) $$
\end{definicao}


\subsection*{Pré-requisitos Acumulados $(\mathcal{R}_{ppc})$}
Um mesmo pré-requisito pode aparecer diversas caminhos de pré-requisitos diferentes, em que isso é causado pelas bifurcações causadas 
por disciplinas que são pré-requisitos para mais de uma disciplina posterior. Isto indica que as disciplinas antes das bifurcações são essenciais 
para a integralização do curso no tempo proposto no PPC. \\

Procurando-se contabilizar a importância de tais disciplinas presentes em diversos caminhos, conta-se a quantidade de pré-requisitos em cada 
caminho do PPC. Sendo $\alpha$ um caminho de pré-requisitos, $||\alpha||$ é número de disciplinas do caminho. Seja $\Gamma$ o conjunto de todos 
os caminhos do PPC, define-se:

\begin{definicao}
O número de pré-requisitos presentes no caminho $\alpha$, denotado por $\mathcal{R}_\alpha$, é

$$ \mathcal{R}_\alpha = ||\alpha|| - 1 $$
\end{definicao}

\begin{definicao}
O número de pré-requisitos acumulados presentes no PPC é
$$ \mathcal{R}_{ppc} = \sum_{\alpha \in \Gamma} \mathcal{R}_\alpha $$
\end{definicao}

\section*{Índice de Complexidade ($\Delta_{ppc}$)}
Pode-se considerar que a quantidade de horas requeridas de estudo por um curso de ensino superior, a composição das disciplinas e seus pré-requisitos, assim 
como a influência que exercem os pré-requisitos na integralização proposto na grade curricular são fatores que contribuem para maior ou menor complexidade 
de um PPC. Sendo assim, define-se: 

\begin{definicao}
O índice de complexidade de um PPC, denotado por $\Delta_{ppc}$, é a soma de turnos efetivos, peso dos pré-requisitos e pré-requisitos acumulados, ou seja,

$$\Delta_{ppc} = \mathcal{T}_{ppc} + \mathcal{P}_{ppc} + \mathcal{R}_{ppc}$$
\end{definicao}

\section*{Índice de Retenção ($\gamma_{ppc}$)}

Cabe à administração do curso a programação semestral das disciplinas do curso, assim como a disposição de cada disciplina na grade de horário semanal. O 
Índice de Complexidade não considera questões como reprovação, oferta de disciplina e tempo de conclusão. Duas hipóteses sobre a administração do curso 
são feitas visando modelar tais questões.

\begin{enumerate}
\item As disciplinas são ofertadas anualmente com semestralidades indicadas no PPC.
\item O aluno terá sucesso ao cursar pela segunda vez uma disciplina. 
\end{enumerate}

Suponha $d$ uma disciplina a qual um aluno, por algum motivo, não realizou sua matrícula ou foi reprovado. Sendo as disciplinas do curso de oferta anual, 
$d$ será deslocada dois semestres a frente. Ademais, considerando os caminhos de pré-requisitos $\beta$ que contém $d$, todas as disciplinas a frente de 
$d$ em $\beta$ também serão deslocadas em dois semestres. Logo, o caminho de pré-requisitos retificado seria na forma:

$$ \beta_{ret}: (m_1, s_1 + 2) \rightarrow (m_2, s_2 + 2) \rightarrow \cdots \rightarrow (m_k, m_k +2) $$

Seja $\Gamma(d)$ o conjunto constituído por os todos os sub-caminhos com início em $d$. Como definido anteriormente, $\mathcal{S}_\beta$ e 
$\mathcal{S}_{\beta_{ret}}$ são os conjuntos dos semestres, assim como $Q_\beta$ e $Q_{\beta_{ret}}$ as respectivas classes de 
equivalência. Observa-se que $Q_{\beta_{ret}}$ é obtido somando-se 2 a cada elemento de $Q_\beta$.

\begin{definicao}
O Índice de Retenção de uma disciplina $d$ em relação a um sub-caminho $\beta \in \Gamma(d)$ é

$$ \gamma_{\beta} = \sum_{j = 1}^{p}(\#Q_{k_j} \log(s_{k_j}+2)) $$
\end{definicao}

\begin{definicao}
O Índice de Retenção de uma disciplina é

$$ \gamma_d = \sum_{\beta \in \Gamma(d)} \gamma_\beta $$
\end{definicao}

\begin{definicao}
O Índice de Retenção de um PPC é

$$ \gamma_{ppc} = \sum_{d} \gamma_d $$
onde o somatório percorre todas as disciplinas do PPC.
\end{definicao}

\section*{Objetivos}

O objetivo geral consiste do desenvolvimento de um software para a análise quantitativa de PPC's. Especificamente, pode-se ressaltar: 

\begin{enumerate}
\item Iniciar um estudante na construção de um aplicativo. (T)
\item Criar um aplicativo que seja útil aos gestores que lidam com o aspecto ensino-aprendizagem na UFCA. (T)
\item Fazer comparações entre PPC's dos cursos da UFCA com de outras instituições de ensino superior. (T)
\item Elaborar linhas de estudo sobre ensino-aprendizagem utilizando possíveis refinamentos da análise estabelecida inicialmente. (P)
\end{enumerate}

\begin{enumerate}

\end{enumerate}

\end{document}
