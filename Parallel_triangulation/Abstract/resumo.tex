\documentclass[a4paper, 12pt]{article}
\usepackage[utf8]{inputenc}
\usepackage[portuges]{babel}
\usepackage{geometry}

\title{Estratégias de Paralelização de Estruturas de Dados Sequenciais}
\date{}
\author{Ikaro Ruan Penha Costa \\
	Prof. Vicente Helano Feitosa Batista Sobrinho}
\begin{document}
\maketitle

Assume-se que os atuais processadores têm chegado aos seus limites de desempenho, o que fomentou o uso da programação concorrente. 
Assim, o presente trabalho tem por objetivo desenvolver algoritmos paralelos a partir de versões sequenciais de estruturas de 
dados de listas encadeadas e de triangulações de Delaunay. Com uso do API OpenMP, inicialmente estudou-se estratégias de 
paralelização da inserção na lista encadeada, a qual o número de elementos a serem inseridos  seria dividido o mais igualmente 
possível entre os threads, núcleos de processamento, para inserção na estrutura de dado. Nesse sentido, cada thread teria uma sublista, 
na qual  os seus respectivos elementos são inseridos, e ao final todas as sublistas seriam conectadas formando a lista resultante. 
Ademais, a operação de busca foi paralelizada semelhantemente, ou seja, cada thread seria responsável por uma seção da lista, o 
que demonstrou resultados diretos na remoção.  Similarmente para a triangulação, um conjunto de pontos seria dividido entre os threads,
para a execução da localização e posterior inserção. Como a inserção na lista e a localização de pontos não necessitam de maiores 
sincronizações, por haver apenas modificação no que seja de propriedade do thread ou apenas leituras na estrutura de dados, 
foi proposta uma estratégia lock-free, sem uso de travas. Isso possibilitou a obtenção de speedups, em uma máquina com quatro 
núcleos de processamento, de 1,20 para inserção de cinco milhões de números inteiros na lista encadeada, de 3,79 para remoção, 
buscando aleatoriamente 20.000 dos seus elementos, e de 2,27 para localização de 30.000 pontos em uma triangulação de Delaunay 
de 100.000 vértices. Contudo, os passos seguintes da inserção na triangulação, construção da região de conflito e atualização,
alteram diretamente a sua estrutura de dados, o que implica no uso de travas para gerir as sincronizações. A partir desse passo,
houve perda de desempenho que gerou resultados insatisfatórios para uma estratégia paralela. Em vista disso, como trabalho futuro 
estuda-se a proposição de outras estratégias de paralelização a serem usadas na inserção incremental em uma triangulação de Delaunay. \\

\textbf{Palavras-chave:} Algoritmos paralelos. Estruturas de dados. Lista encadeada. Triangulação de Delaunay. \\

\end{document}
