\documentclass[a4paper, 12pt]{article}
\usepackage{geometry}
\usepackage[utf8]{inputenc}
\usepackage[portuges]{babel}
%\usepackage{hyphenat}
\usepackage{ragged2e}
\usepackage{setspace}
\usepackage{pgfplots}
\usepackage{subcaption}
\usepackage{graphicx}
\usepackage{tabularx}


%\setstretch{1.5}
%\hyphenation{ar-ma-ze-na-gem}
%\hyphenation{mo-der-nos}
%\hyphenation{po-de-rão}
%\hyphenation{ci-en-tí-fi-cos}
%\hyphenation{o-pe-ra-çõ-es}

\title{ Relatório Final de Projeto
	%\huge Algoritmos Paralelos para Triangulações de Delaunay de Ordem Superior 
	%     Aplicados na Modelagem de Terrenos
	}
\date{}
\author{ Vicente Helano Feitosa Batista Sobrinho \\
	\texttt{vicente.sobrinho@ufca.edu.br}
	\and
	Ikaro Ruan Penha Costa \\
	\texttt{ikaroruan@outlook.com}
	}

\begin{document}
\maketitle

\section*{Identificação do Projeto}
\textbf{Título} \\
Algoritmos paralelos para triangulações de Delaunay de ordem superior aplicados na modelagem de terrenos \\ \\
\textbf{Edital de Referência} \\
PIBIC Edital Nº02/2016

\section*{Área do Conhecimento Predominante}
Ciência da Computação

\section*{Resumo}
Diversos problemas científicos recaem no uso de geração de malhas para obtenção de uma modelagem matemática computacional. Dentre 
essas malhas, as Triangulações de Delaunay são as mais covenientemente usadas devido suas caracteríscas da qualidade 
dos triângulos formados. Muitos problemas modernos envolvem variadas superfícies e formas a serem modeladas, assim 
como exigem maior acurácia em seus cálculos, o que requer maior desempenho computacional. Desse modo, uma alternativa 
relacionada à melhora na performance de tais algoritmos consiste na Computação Paralela. Além disso, códigos 
em paralelo, por ventura, necessitam de sincronizações para evitar possíveis erros lógicos, o que implica no estudo
das estratégias de travas mais adequadas para a estrura de dados que armazena a triangulação. Portanto, 
este trabalho tem o objetivo de produzir implementações paralelas de triangulações de Delaunay, de modo a permitir 
avanços no desempenho de suas construções. \\

\textbf{Palavras-chave:} Triangulação de Delaunay, algoritmos paralelos, estratégia de travas. \\

\section*{Introdução}

% Citar isso mesmo??
% TODO: Melhorar essa Introdução

Dentre as malhas triangulares, as triangulações de Delaunay representam aquelas que possuem triângulos de boa qualidade  
em seu interior, ou seja, configuram-se por ângulos internos maximizados, e de menor qualidade no entorno de sua fronteira. 
Isso dá-se pela propriedade do circuncírculo vazio, em que garante que o circuncírculo formado por qualquer triângulo de uma 
triangulação \textit{T} não conterá outro ponto de \textit{T} em seu interior. \\

Por conseguinte, mesmo podendo o algoritmo de Bowyer-Watson \cite{bowyer, watson} construir uma triangulação de Delaunay em $O(n logn)$, 
algumas ocasiões podem demandar elevados tempos de execução, a depender da quantidade e da disposição espacial de seus pontos. Além disso, 
considerando a conjuntura atual de microprocessadores, em que se pode ter chegado a um limite de desempenho, percebe-se a 
necessidade do uso de \textit{Computação Paralela} para construção de triangulações com  alto desempenho. A partir do 
paralelismo é possível repartir diferentes operações entre \textit{threads} (unidades de processamento), com o intuito de obter 
reduções nos tempos de execução de uma triangulação. A essas reduções dá-se o nome de \textit{speedup}. \\

Além disso, o algoritmo de inserção proposto por Bowyer-Watson configura-se por uma adição incremental de pontos. Ao inserir um ponto $P$ 
na triangulação, este poderá estar em desacordo com a propriedade do circuncírculo. Logo, ao conjunto de faces de $T$ em que $P$ viola tal 
propriedade, dá-se o nome de \textit{região de conflito}. Essa região compõe-se das faces que sofrerão alterações ao longo 
da inserção de $P$. \\

Contudo, considerando a paralelização da inserção, diferentes threads não poderão modificar mesmas regiões da triangulação, pois,
como não se pode ter certeza da ordem de execução dos threads, alterações concorrentes de um mesmo elemento (face ou vértice)  
podem ocasionar em resultados divergentes da correta triangulação a ser formada. Consequentemente, necessita-se de um modelo de 
sincronização baseado em travas, que garantem a modificação ou o acesso único de threads por vez. A partir disso, nota-se 
também a necessidade do estudo de mecanismos de travas que sejam capazes de oferecer speedups eficientemente de acordo com a 
implementação paralela em questão. \\



\section*{Justificativa}
\justify
Devido à grande aplicabilidade de triangulações de Delaunay em problemas científicos, especialmente na modelagem de terrenos,
torna-se viável o estudo de técnicas de suas contruções de modo a garantir melhor desempenho. Desse modo, estudos que exigem maior 
precisão em seus dados demandarão maior poder de processamento, de modo a refletir no tempo de execução do processo de construção da triangulação.
Nesse sentido, considerando as atuais limitações no desempenho de processadores, é importante o estudo de técnicas de paralelização 
da contrução de triangulações de Delaunay com a finalidade de obter melhores desempenhos a partir de microprocessadores 
com memória compartilhada. \\

\section*{Referencial Teórico}
Estudos tem demonstrado bons resultados quanto à paralelização de algoritmos para problemas geométricos, incluindo triangulações de 
Delaunay. Nesse sentido, Batista et al. \cite{batista} demonstrou a possibilidade de paralelização da inserção incremental a partir 
da \textsc{cgal}, assim utilizando métodos que permitam maior acesso concorrente de faces na localização, dentre eles o de 
travas melhoradas. Já Kohout \cite{kohout} apresenta tanto métodos de inserção paralela diferentes como de travamento e oference, ainda,
maiores detalhes em sua tese de doutorado \cite{kohoutthesis}. Cignoni et al. \cite{cignoni} apresentou resultados favoráveis para
triangulações de Delaunay tridimensionais a partir da construção incremental. \\

Vale ressaltar que o \texttt{omp\_lock\_t} foi a trava usada em OpenMP, seguindo especificações apresentadas pelo OpenMP Review Board
\cite{openmp_manual} e sugestões de uso de Chapman et al. \cite{chapman}. Enquanto o \texttt{std::mutex} é a trava usada para implementação
em C++ Threads, a partir do sugerido por Williams \cite{williams}. \\


\section*{Objetivos}
O objetivo consiste da implementação de algoritmos paralelos para construção de triangulações de Delaunay. Especificamente dispostos
da seguinte maneira:

\begin{itemize}
	\item Determinar qual plataforma de desenvolvimento é mais adequada para esta situação: OpenMP, Intel TBB ou C++ Threads. (T)
	\item Implementar os algoritmos paralelos de construção de triangulações de Delaunay de ordem superior usando a 
		\textit{Computational Geometry Algorithm Library} - \textsc{cgal}. (P)
\end{itemize}

\section*{Metodologia}
Inicialmente, foi implementado um algoritmo sequencial para construção de triangulações de Delaunay. Esse foi desenvolvido através 
de uma estrutura de dados baseada em faces, a qual armazena vértices e faces, assim como cada vértice possui informação de uma 
face incidente e cada face informações sobre suas vizinhas. A técnica de inserção empregada foi baseada em Bowyer-Watson, chamada
de \textit{inserção incremental}, que foi dividida em três fases: localização, busca da região de conflito e atualização. A 
localização consiste da  verificação da posição de um ponto $P$ a ser inserido na triangulação. Assim, de posse da localização, pode-se buscar 
quais faces terão $P$ no interior seu circuncírculo associado, o que implica na construção da região de conflito. Em seguida, as faces 
em conflito formarão uma cavidade na qual será inserido $P$ e novas faces serão formadas, consistindo da fase de atualização da triangulação. \\

Os predicados geométricos para verificação de orientações de pontos
e teste do circuncírculo vazio são provenientes da implementação de Shewchuk \cite{shewchuk}. Quanto ao armazenamento da triangulação, 
foi utilizado o \textit{Compact Container} e, para melhor eficiência na inserção incremental, a ordenação espacial dos pontos a partir
 do \textit{Spatial Sort}, ambos da \textsc{cgal} \cite{cgal}. \\

Seguidamente, havia a necessidade de escolher a plataforma de desenvolvimento paralelo para essa situação. Assim, iniciou-se com 
a implementação de uma estrutura de dados em \textit{lista encadeada} com inserção concorrente usando OpenMP, Intel TBB, 
Intel CilkPlus e C++ Threads. De modo geral, tanto OpenMP quanto C++ Threads ofereceram maior liberdade para implementações
paralelas, assim como travas próprias e ferramentas associadas às particularidades de suas plataformas. \\

Nesse sentido, a ideia principal é de dividir os pontos a serem inseridos em conjuntos disjuntos, denominados \textit{workloads}, 
de acesso exclusico por thread. Cada thread executará todos os passos da inserção (localização, região de conflito e atualização) até
que tenham inserido todos os pontos de seus respectivos workloads. Entretanto, a necessidade de sincronização surge com a possibilidade
de modificações simultâneas de uma mesmas regiões de conflito. \\

Para isso, estratégias de travas foram estudadas para escolha da melhor forma de sincronização para inserção de pontos na triangulação. 
Assim, possuindo cada vértice um \textit{lock} (trava) associado, quando uma face tiver seus três vértices travados, será considerada 
travada. Dessa forma, consideram-se as seguintes estratégias: \\

\begin{itemize}
	\item Trava Simples: para leitura ou modificação de faces, deve-se travar todos os vértices de uma face.
	\item Trava Simples com Prioridades: cada thread terá uma prioridade para travar os vértices. Caso um vértice esteja travado, 
		o thread de maior prioridade deverá esperar até adquirir propriedade sobre ele, mas se for de menor prioridade estará
		inapto para travamento, devendo desistir da inserção de $P$ momentaneamente e voltar a inseri-lo posteriormente.
	\item Trava Melhorada: consiste em travar apenas um vértice da face para operações de leitura (para a localização) e 
		todos os vértices da face para modificação (região de conflito e consequente atualização).
\end{itemize}

Deve-se notar que a Trava Simples com Prioridade foi baseada em Kohout \cite{kohout}, apesar de não ter tido o comportamento esperado 
no algortimo aqui implementado, logo não foi considerado. Enquanto a Trava Melhorada em Batista \cite{batista}, que permite, assim, leitura simultânea
durante a fase de localização. \\

Dessa forma, torna-se importante a análise da eficiência da implementação paralela, de acordo com as estratégias de travas utilizadas,
visando a obtenção de speedups.

\section*{Resultados e Discussão}

De início, as plataformas foram escolhidas foram OpenMP e C++ Threads por permitirem maior liberdade de implementação e maior número de
ferramentas, incluindo de sincronizações. Para os teste de execução foram usadas distribuições uniformes de pontos no espaço em quantidade $2^k$, 
em que $k$ varia entre $10$ e $20$, em uma máquina Dell com processador Intel Core i7-4510U, clock máximo 3,10 GHz, cache de 4MB e 4 CPUs.
Nesse contexto, pode-se verificar na Figura \ref{exec:simples}, a qual apresenta resultados para estratégias de travas simples em 
paralelo com quatro threads, que os resultados não foram satisfatórios, haja vista que não foram obtidos speedups. \\

\begin{center}
\begin{tikzpicture}
  \begin{axis}[
  xlabel = {$k$, para $2^k$},
  ylabel = {Tempo de execução (s)},
  xmin = 15, xmax = 20,
  ymin = 0, ymax = 300,
  xtick = {},
  ytick = {0, 50, 100, 150, 200, 250, 300},
  legend pos = outer north east,
  ymajorgrids = true,
  grid style = dashed,
  ]

  \addplot[color = blue, mark = square]
  coordinates{(10, 0.02118)(11, 0.04764)(12, 0.11615)(13, 0.27277)(14, 0.68783)(15, 1.45394)(16, 4.05375)(17, 12.30595)(18, 29.69340)
  	(19, 73.94110)(20, 170.62250)};
    
  \addplot[color = red, mark = x]
  coordinates{(10, 0.02064)(11, 0.04538)(12, 0.10083)(13, 0.21762)(14, 0.55693)(15, 1.52596)(16, 4.03057)(17, 11.566)(18, 31.07105)
  	(19, 94.87275)(20, 224.77750)};

  \addplot[color = green, mark = star]
  coordinates{(10, 0.02273)(11, 0.04792)(12, 0.10017)(13, 0.27375)(14, 0.53642)(15, 1.53473)(16, 4.52265)(17, 11.83385)(18, 30.77380)
  	(19, 80.61200)(20, 296.27150)};
  \legend{Sequencial, OpenMP, C++ Threads}

  \end{axis}
\end{tikzpicture}
\captionof{figure}{Tempos de execução para implementação senquencial e paralela com travas simples.}
\label{exec:simples}
\end{center}

Contudo, a partir dos experimentos pode-se perceber que a localização foi a operação que consumiu maior tempo de processamento, como
pode ser visto na Figura \ref{exec:parcelas} a qual mostra o consumo de tempo pelas operações de inserção (localização, construção da região de
conflito e atualização).

\begin{center}
\begin{tikzpicture}
  \begin{axis}[ybar stacked,
  xmin = 18, xmax = 20,
  ymin = 0, ymax = 300,
  xlabel = {$k$, em que $2^k$ pontos},
  ylabel = {Tempo de execução (s)},
  xtick = {18, 19, 20},
  legend pos = outer north east,
  ymajorgrids = true,
  grid style = dashed,
  enlargelimits=0.15,
  ]

  \addplot+[ybar]
  coordinates{(18, 34.46305)(19, 69.64040)(20, 188.97050)};
  \addplot+[ybar]
  coordinates{(18, 2.93920)(19, 5.55268)(20, 11.65645)};
  \addplot+[ybar]
  coordinates{(18, 5.012505)(19, 8.02254)(20, 31.70463)};
  \legend{Localização, Região de Conflito, Atualização}
  \end{axis}
\end{tikzpicture}
\captionof{figure}{Parcelas de execução de cada etapa da inserção.}
\label{exec:parcelas}
\end{center}

Diante disso, percebeu-se que a Trava Simples gerou maiores sincronizações, pois cada thread obtém acesso exclusivo para leitura
das faces no instante de localização, sendo que um thread não poderia travar uma face cujos locks já estivessem em posse de outro thread.
Caso isso ocorra, o algoritmo desistirá momentaneamente da inserção daquele ponto, a isso intitula-se de \textit{retreat}, e nova 
tentativa será feita posteriormente. \\ 

Logo, a partir dos dados mostrados pela Figura \ref{retreats:openmp} percebe-se um alto número de retreats ocorrendo na localização e baixo número
para região de conflito. Contudo, a localização apenas utiliza-se de leituras dos vértices e faces, o que permite múltiplas leituras com segurança
pelos threads, com exceção daqueles travados pela região de conflito. Além disso, deve-se ressaltar que como threads não poderão 
modificar mesmas faces da triangulação, a construção da região de conflito também poderá ocasionar retreats.\\

A partir disso, nota-se a necessidade do uso de Travas Melhoradas, que permitem menores sincronizações no processo de localização.
Desse modo, a Figura \ref{retreats:openmp} mostra que esta estratégia permitiu uma conveniente redução no número de retreats pela localização.
Ademais, quanto à região de conflito houve uma leve redução, mas não considerável. \\

\begin{center}
\begin{tikzpicture}
  \begin{axis}[
  xlabel = {$k$, para $2^k$},
  ylabel = {Número de Retreats},
  xmin = 15, xmax = 20,
  ymin = 0, ymax = 4000,
  xtick = {},
  ytick = {250, 500, 1000, 2000, 3000, 4000},
  legend pos = outer north east,
  ymajorgrids = true,
  grid style = dashed,
  ]
  \addplot[color = blue, mark = triangle]
  coordinates{(10, 1085)(11, 810)(12, 260)(13, 628)(14, 1099)(15, 1604)(16, 1190)(17, 1824)(18, 1730)(19, 3936)(20, 2368)};

  \addplot[color = red, mark = x]
  coordinates{(10, 177)(11, 424)(12, 264)(13, 156)(14, 606)(15, 423)(16, 487)(17, 248)(18, 470)(19, 222)(20, 124)};
  \legend{Trava Simples, Trava Melhorada}
  \end{axis}
\end{tikzpicture}
\captionof{figure}{Número de retreats na localização para execução paralela em OpenMP com 4 threads.}
\label{retreats:openmp}
\end{center}

Apesar da redução no número de retreats, não houveram melhoras no tempo de execução dos algoritmos paralelos. Apenas a implementação
usando C++ Threads apresentou pequena melhora, mas não chegando próximo de um speedup. Tais dados podem ser
verificados nas Figuras \ref{exec:improved} e \ref{exec:travas}.

\begin{center}
\begin{tikzpicture}
  \begin{axis}[
  xlabel = {$k$, para $2^k$},
  ylabel = {Tempo de execução (s)},
  xmin = 15, xmax = 20,
  ymin = 0, ymax = 300,
  xtick = {},
  ytick = {0, 50, 100, 150, 200, 250, 300},
  legend pos = outer north east,
  ymajorgrids = true,
  grid style = dashed,
  ]

  \addplot[color = blue, mark = square]
  coordinates{(10, 0.02118)(11, 0.04764)(12, 0.11615)(13, 0.27277)(14, 0.68783)(15, 1.45394)(16, 4.05375)(17, 12.30595)(18, 29.69340)
  	(19, 73.94110)(20, 170.62250)};
  \addplot[color = red, mark = x]
  coordinates{(10, 0.01982)(11, 0.04514)(12, 0.10338)(13, 0.24688)(14, 0.65469)(15, 1.81986)(16, 4.86161)
  	(17, 11.13360)(18, 44.55410)(19, 87.34675)(20, 240.88950)};
  \addplot[color = green, mark = star]
  coordinates{(10, 0.02228)(11, 0.04722)(12, 0.09558)(13, 0.26587)(14, 0.53982)(15, 1.52206)(16, 4.42327)(17, 11.69265)
  	(18, 30.84685)(19, 79.68945)(20, 267.64950)};
  \legend{Sequencial, OpenMP, C++ Threads}
  \end{axis}


\end{tikzpicture}
\captionof{figure}{Tempo de execução dos algoritmos sequencial e paralelos com travas melhoradas e 4 threads.}
\label{exec:improved}
\end{center}

\begin{center}
\begin{tabular}{rl}
\begin{tikzpicture}[scale=0.8]
  \begin{axis}[
  title = OpenMP,
  xlabel = {$k$, para $2^k$},
  ylabel = {Tempo de execução (s)},
  xmin = 15, xmax = 20,
  ymin = 0, ymax = 300,
  xtick = {},
  ytick = {0, 50, 100, 150, 200, 250, 300},
  legend pos = north west,
  ymajorgrids = true,
  grid style = dashed,]
  \addplot[color = blue, mark = triangle]
  coordinates{(10, 0.02064)(11, 0.04538)(12, 0.10083)(13, 0.21762)(14, 0.55693)(15, 1.52596)(16, 4.03057)(17, 11.566)(18, 31.07105)
  	(19, 94.87275)(20, 224.77750)};
  \addplot[color = red, mark = square*]
  coordinates{(10, 0.01982)(11, 0.04514)(12, 0.10338)(13, 0.24688)(14, 0.65469)(15, 1.81986)(16, 4.86161)
  	(17, 11.13360)(18, 44.55410)(19, 87.34675)(20, 240.88950)};
  \legend{Trava Simples, Trava Melhorada}
  \end{axis}
\end{tikzpicture}

&

\begin{tikzpicture}[scale=0.8]
  \begin{axis}[
  title = C++ Threads,
  xlabel = {$k$, para $2^k$},
  ylabel = {Tempo de execução (s)},
  xmin = 15, xmax = 20,
  ymin = 0, ymax = 300,
  xtick = {},
  ytick = {0, 50, 100, 150, 200, 250, 300},
  legend pos = north west,
  ymajorgrids = true,
  grid style = dashed,]
  \addplot[color = blue, mark = triangle]
  coordinates{(10, 0.02273)(11, 0.04792)(12, 0.10017)(13, 0.27375)(14, 0.53642)(15, 1.53473)(16, 4.52265)(17, 11.83385)(18, 30.77380)
  	(19, 80.61200)(20, 296.27150)};
  \addplot[color = red, mark = square*]
  coordinates{(10, 0.02228)(11, 0.04722)(12, 0.09558)(13, 0.26587)(14, 0.53982)(15, 1.52206)(16, 4.42327)(17, 11.69265)
  	(18, 30.84685)(19, 79.68945)(20, 267.64950)};
  \legend{Trava Simples, Trava Melhorada}
  \end{axis}
\end{tikzpicture}

\end{tabular}
\captionof{figure}{Comparação dos tempos de execução de Travas Simples e Melhoradas. \\}
\label{exec:travas}
\end{center}

Nesse sentido, embora melhora no número de retreats, não houve significativa redução no tempo de execução dos algoritmos. Assim, 
o maior consumo de tempo continua sendo pela localização de pontos. \\

\section*{Conclusão}
Ainda que seja uma ferramenta muito útil para ganhos em desempenho computacional, a Computação Paralela necessita diversos estudos 
e elaborações de estratégias para se obter os melhores resultados. Desse modo, aplicando-a em contruções de triangulações de Delaunay
tem sido tema recorrente de estudo atualmante, havendo ainda resultados positivos. Entretanto, apesar das estratégias de sincronização
terem performance considerável no gerenciamento de sincronizações, dificuldades na obtenção de tempos de execução foram enfrentadas. \\

Portanto, haja vista os resultados produzidos por esse trabalho, nota-se a necessidade de uma análisa mais ampla das possívies causas
de lentidão geradas nos algoritmos paralelos. Logo, pode-se considerar para o futuro a verifição dos possíveis causadores de menor 
desempenho na localização de pontos como também o estudo estratégias de travas adequadas ao método de inserção com o intuito de obteção 
de speedups. \\


%\newpage
\bibliographystyle{unsrt}
\bibliography{references}
\end{document}
